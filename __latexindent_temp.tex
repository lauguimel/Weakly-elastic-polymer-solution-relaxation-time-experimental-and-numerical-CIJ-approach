\documentclass[preprint,3p,times]{elsarticle}


%% The amssymb package provides various useful mathematical symbols
\usepackage{amssymb}
%% The amsthm package provides extended theorem environments
\usepackage{amsthm}
\usepackage{epstopdf}
\usepackage{graphics,graphicx,amsbsy,amssymb}
\usepackage{float}
\usepackage{import} 
\usepackage{color} 

%% The lineno packages adds line numbers. Start line numbering with
%% \begin{linenumbers}, end it with \end{linenumbers}. Or switch it on
%% for the whole article with \linenumbers.
%% \usepackage{lineno}

\journal{}

\begin{document}

\begin{frontmatter}

\title{Weakly elastic industrial ink: an experimental and numerical study for relaxation time determination}

\author[1]{Guillaume Ma\^itrejean\corref{cor1}}
\ead{guillaume.maitrejean@univ-grenoble-alpes.fr}

\author[1]{Maxime Rosello}
\author[1]{Denis CD Roux}
\author[1]{Pascal Jay}
\author[2]{Bruno Barbet}
\author[2]{Jean Xing}

\cortext[cor1]{Corresponding author}

\address[1]{Laboratoire Rh\'éologie et Proc\'ed\'es, Univ. Grenoble Alpes, LRP, F-38000 Grenoble, France CNRS, LRP, F-38000 Grenoble, France }
\address[2]{Markem-Imaje Industries, ZA de l'Armailler 9, rue Gaspard Monge - BP 110 26501 Bourg-L\'és-Valence - France}

\begin{abstract}
    The relaxation time of a weakly elastic polymer solution is measured by comparing numerical and experimental drop shapes. Computations use a viscoelastic Oldroyd-B model, the relaxation time of which is fitted using experimental results. This numerical approach is particularly convenient for weakly elastic solution physical parameters of which can not be measured by experimental rheometry.
\end{abstract}

\begin{keyword}
\end{keyword}

\end{frontmatter}

%% \linenumbers

%% main text
\section{Introduction}
Capillary breakup phenomena have a wide range of applications from inkjet printing to DNA sampling. In micro jetting devices, polymer solutions often experience non
Newtonian behaviour which greatly influence breakup dynamics. Drops generation from non Newtonian fluid jets breakup is a well known topic which has already been
addressed both numerically and experimentally \cite{morrison2011inkjet,rodriguez2015experimental,mcilroy2013modelling}. Elastic and viscous effects are known to have a great influence on the breakup dynamics. More precisely, they
are known to delay the onset of the jet breakup \cite{rayleigh1892xvi, gordon1973instability}. In the present work, the relaxation time of a weakly elastic ink used in industrial continuous inkjet printing (CIJ) devices is determinated. This ink is a low viscosity dilute polymer with high polydispersity. A first estimation of the elastic relaxation time is calculated using Zimm theory \cite{zimm1956dynamics}, and is found to be out of the measurement range of both extensional rheometry (ROJER \cite{keshavarz2015studying}) or microfluidic devices \cite{galindo2013microdevices}. As a result, an original approach is introduced to determine the relaxation time, relying on the comparison of capillary breakup shapes between numerical computations and experiments.

\section{Experimental setup}
The experimental device is similar to the one used for ROJER extensional rheometry measurements \cite{rodriguez2015experimental} (see figure 1). The flow is generated by a pump and the jet is created using a micro-nozzle. Then, it is strobed at given frequency synchronized with the drive frequency ($10 kHz < f_d < 100 kHz$) in order to display a static image ($1024\times778$ pixels with $1 px \approx 1 \mu m$ ). The visualization software is ImageXpert.

\begin{figure}[H]
    \centering
    \includegraphics[width=12cm]{device.png}
    \caption{}
    \label{device}
\end{figure}

Jetting conditions are chosen to ensure a constant dimensionless wave number $x=0.6$ for all nozzles and all inks. $x$ is calculated with the wave length $\lambda$ and made dimensionless with the unperturbed jet radius $R_0$ as follow :

\begin{equation}
  x=\frac{2 \pi R_0}{\lambda}=\frac{2 \pi R_0 f}{v},
\end{equation} 

with $f$ the disturbance frequency and $v$ the undisturbed jet average velocity. Reynolds and Ohnesorgue numbers, respectively $\rho U R_0 / \eta_0$ and $\eta_0/\sqrt{\rho R_0 \sigma}$ are calculated for undisturbed jets : $Re=110$ and $Oh=0.2$. These numbers are constant for every disturbance amplitudes.

Jets obtained for small disturbance amplitudes, i.e. $A_t \in [2 \, V,13 \, V]$, are  depicted Figure \ref{BUlinear}. We can see that breakup dynamics experience linear evolution and that small threads linking main drops to satellites (for instance $A_t = 13 \, V$) can be observed. These threads are typical of those observed during the breakup of weakly elastic polymer strain hardening solutions \cite{christanti2002effect}.

\begin{figure}[H] 
    \def\svgwidth{7cm} 
    \vspace{2cm}
    \centering
     \import{}{k06_5127_linear.eps_tex}
     \caption{Breakup shapes obtained in the linear regime ($A_t \in [2 \, V,13.2 \, V]$) for $x=0.6$.}
      \label{BUlinear}
  \end{figure}
 
The latter observation of viscoelasticity of the ink is consistent with its molecular composition. Indeed the ink is in the semi-dilute regime and it presents quite flexible polymer chains likely to exhibit an elastic behavior. In the next en extensive experimental study is performed to accurately determine the rheological characteristics of the ink.

\section{Rheological characterization}
Both the density $\rho = 873 kg.m^{-3}$ and the surface tension $\sigma = 22.8 mN.m^{-1}$ of the fluid have been carefully measured.

\subsection{Shear viscosity}
The shear viscosity is measured using three devices covering a wide range of shear rates, from $1s^{-1}$ to $10^6 s^{-1}$:
\begin{itemize}
    \item ARG-2 from TA-Instruments for low shear rates (from $10^0 s^{-1}$ to $10^3 s^{-1}$),
    \item a piezo-rheometer developed by \citet{buchanan2005high} for medium range shear rates (from $10^1 s^{-1}$ to $10^5 s^{-1}$),
    \item m-VROC from RheoSense for high shear rates (from $10^4 s^{-1}$ to $10^6 s^{-1}$)
\end{itemize}
As we can see on \ref{beahaviorLaw}, the viscosity can be fitted using a Cross model depicting the shear-thinning behavior:
\begin{equation}
    \eta(\dot{\gamma})=\eta_{\infty} + \frac{\eta_0 - \eta_{\infty}}{(1+\tau_{\dot{\gamma}} \dot{\gamma})^{n}},
    \label{crossEq}
  \end{equation}
where $\tau_{\dot{\gamma}}=2.10^{-7}s$ is the natural time (i.e., inverse of the shear rate at which the fluid changes from Newtonian to power-law behavior), $n=0.72$ the flow behavior index, and where finally $\eta_0=4.7 mPa.s $ and $\eta_\infty = 1 Pa.s$ are respectively the upper and lower limits of the power-law.

\begin{figure}[H]
    \centering
    \includegraphics[width=12cm]{behaviorLaw.png}
    \caption{}
    \label{beahaviorLaw}
\end{figure}

\subsection{Viscoelasticity}
\subsubsection*{Theoritical approach}
Elastic phenomena are characterized by stresses relaxation in the fluid. The key parameter to describe them is the polymer chain relaxation time $\lambda_p$. There is a finite number of relaxation times for each polymer in solution, each corresponding to a mode of polymer relaxation \cite{de1979scaling}. For modelling the behaviour of polymer solutions, one used to consider only the longest relaxation time corresponding to the first relaxation mode. Several microscopic analyses have been developed to calculate it according to the properties of the polymer chains.

The first theory on the subject is established by Rouse in 1953 \cite{rouse1953theory}.  The latter models the polymer chains as a succession of fixed stiffness springs and accounts only for the friction interactions between the chains. The longest relaxation time determined by this theory is rated $\lambda_R$ and is calculated via the relationship:

\begin{equation}
    \lambda_R = \frac{6 \eta_0 M}{\pi^2 \rho R T},
    \label{rouse} 
    \end{equation}

\bibliographystyle{plainnat}
\bibliography{biblio}
\end{document}
